\documentclass[a4paper,12pt,times,numbered,print,index]{report}
\usepackage[english]{babel}
\usepackage[utf8]{inputenc}

\oddsidemargin  0.01mm       % USA 5mm?
\evensidemargin 0.01mm       % USA 5mm?
\headheight -0.03mm             % 10mm ok
\headsep  -0.03mm
\hoffset -3mm
% was commented out before
\textheight 250mm            % USA 240mm?
\textwidth 180mm             % USA 160mm?
\topmargin -10mm           % before 18/5/93 this was -20mm
\topskip -10mm

\renewcommand{\baselinestretch}{1.40}

\renewcommand {\arraystretch}{1.20}

\usepackage{dcolumn}
\usepackage{amsmath}
\usepackage{amssymb}
\usepackage{graphicx}
\usepackage{pdfpages}
\usepackage{fullpage}
\usepackage{caption}
\usepackage{float}
\usepackage{fancyhdr}
\usepackage{amsmath}
\usepackage{multirow}
\usepackage{commath}
\usepackage{natbib}
\setcitestyle{aysep={,}}
%\usepackage{apacite}
%\usepackage{biblatex}
\usepackage{mathtools}
\usepackage{subcaption}
\usepackage{mathrsfs,amssymb}
\usepackage{amsthm}
\usepackage{enumitem}
\usepackage{booktabs}
\usepackage{verbatim}
\usepackage{enumitem,kantlipsum}
\usepackage{chngcntr}
\usepackage{apptools}
\usepackage{adjustbox}
\usepackage{epstopdf}
\usepackage{pdflscape}
\usepackage{afterpage}
\usepackage[pdftex,colorlinks=true,linkcolor=cyan,linktoc=all]{hyperref}
\usepackage{siunitx}
\usepackage{authblk}
\usepackage{soul}

\usepackage{titletoc}% http://ctan.org/pkg/titletoc
\titlecontents*{chapter}% <section-type>
[0pt]% <left>
{}% <above-code>
{\bfseries\chaptername\ \thecontentslabel\quad}% <numbered-entry-format>
{}% <numberless-entry-format>
{\bfseries\hfill\contentspage}% <filler-page-format>

\usepackage[font=footnotesize,labelfont=bf]{caption}
\captionsetup[sub]{font=scriptsize,labelfont=bf}

\renewcommand{\sectionautorefname}{Section}
\renewcommand{\subsectionautorefname}{Section}
\renewcommand\thesection{\arabic{section}}

\interfootnotelinepenalty=10000
\setcitestyle{aysep={,}}
\newcolumntype{L}{@{}l@{}}
\newcommand{\mc}[1]{\multicolumn{1}{c}{#1}}
\sisetup{table-space-text-post = **}
\definecolor{darkblue}{rgb}{0,0,.6}
\hypersetup{citecolor=darkblue,linkcolor=darkblue,urlcolor=darkblue}

\makeatletter

\def\@seccntformat#1{\@ifundefined{#1@cntformat}%
	{\csname the#1\endcsname\quad}  % default
	{\csname #1@cntformat\endcsname}% enable individual control
}
\let\oldappendix\appendix %% save current definition of \appendix
\renewcommand\appendix{%
	\oldappendix
	\newcommand{\section@cntformat}{\appendixname~\thesection\quad}
}
\makeatother
\usepackage{cleveref} % just for this example

\numberwithin{equation}{section}
\newtheorem{theorem}{Theorem}[section]
\DeclareMathOperator*{\argmin}{argmin}
\newtheorem{remark}{Remark}[section]
\newtheorem{assumption}{Assumption}
\allowdisplaybreaks
\newtheorem{lemma}{Lemma}

\newcommand{\assumptionautorefname}{Assumption}
\newcommand{\lemmaautorefname}{Lemma}
\newcommand{\qedd}{\tag*{$\qed$}}

\pagestyle{fancy}
\fancyhf{}
\renewcommand{\headrulewidth}{0pt}
\cfoot{\thepage}

\title{{Parametric Single-Index Models: Simulation and Empirical Results} \vspace{2cm}}
\author{Ying Zhou}
\date{March \\ 2021}

\begin{document}
\begin{titlepage}
	\begin{center}
		\vspace*{1cm}
			
		\Huge
		\textbf{Parametric Single-Index Models: Simulation and Empirical Results}
			
		\vspace{2.5cm}
			
		\large
		\textbf{Ying Zhou} \\
			
		\vspace{0.5cm}
			
		Supervised by: \\
		Professor Jiti Gao \\
		Dr Hsein Kew
			
		\vfill
			
		A thesis presented for PhD progress review.
			
		\vspace{0.8cm}
			
		\includegraphics[width=0.4\textwidth]{plots/monash-university-logo.png}
			
		Department of Econometrics and Business Statistics \\
			%		Monash University\\
			
		March,  2021
			
	\end{center}
\end{titlepage}
	
\pagenumbering{arabic}
	
\section{Overview}
Accurate forecasting of economic and financial variables is a challenging task in time-series analysis because the real world data have different characteristics. Simply using a transformed version of the original data (for example, taking the first order difference) may ignore some key features involved in the original data.

Linear models are the most commonly used model in economics and finance as it is simple to implement and easy to interpret. However, when integrated time-series are included, the findings concluded from linear models may suffer from potential misbalancing or spurious problem (\cite{phillips2015halbert}). Therefore, econometricians have created various models and estimation methodologies, and a vast number of studies have been published accordingly. The main objective of this thesis is to develop models and methods to deal with these problems cause by predictors with various features. 

The spurious problem occurs when variables in the regression are highly persistent. As in the study of \cite{whittington1990fertility}, which adopts a linear model to investigate the effect that tax benefits have on general fertility rate. However, since both tax benefits and general fertility rate are highly persistent, the model actually shows a misleading statistical evidence of linear relationship.\cite{crump2011fertility} revisits their study and they find no evidence of a linear co-integrating relationship between fertility and personal exemption. Thus, we considers a
semi-parametric nonlinear model with a cointegrated system. 

This model extends the fully non-parametric time-varying coefficients models developed in \cite{li2016estimation}  by adding non time-varying coefficients. This is a useful extension because it allows for the inclusion of dummy variables and deterministic time trends in the regression. We develop a multi-step estimation procedure that can be employed to estimate both the time-varying coefficients non-parametrically and the non-time-varying coefficients parametrically. We apply the proposed model to
study the impact of tax incentives on fertility. Our main findings include:

\begin{enumerate}
	\item The time-varying coefficients model suggests a nonlinear co-integrating relationship between tax benefits and general fertility rate.
	
	\item The nonlinear co-integrating relationship between tax benefits and fertility rate has weakened considerably over time.
	
	\item Even though new tax incentives have been introduced, they are not effective in increasing fertility rate any more.
\end{enumerate}

By using the semi-parametric time-varying model, we avoid the spurious problem and uncover the nonlinear co-integrating relationship between the dependent variables and independent variables. 

Another challenge in using linear predictive model is the misbalancing problem , which happens when some of the predictors have long memory and the dependent variable has short memory. (\cite{phillips2015halbert}). Stock return prediction is a typical case because stock return is stationary while most of the predictors used by previous studies are non-stationary (\cite{campbell1988dividend}, \cite{fama1990stock} and \cite{pesaran1995predictability}). In-sample, 
a large number of studies find evidence of predictability. Out-of-sample, little consensus exists on the question of whether stock return is predictable and which variables provide a better forecasting performance (\cite{welch2008comprehensive}).

To achieve a better out-of-sample performance, some researchers consider nonlinear models to capture the nonlinear relationship between stock return and its predictors (\cite{park2001nonlinear} and \cite{park2002nonstationary}). Our study extends the nonlinear model with a single integrated regressor developed in \cite{park2001nonlinear} to allow for multiple integrated regressors. The nonlinear we propose is given below:
\begin{equation}
y_{t}=f\left( x_{t-1}^{\prime }\theta _{0},\gamma _{0}\right) +e_{t},\ \ \
t=2,...,T, 
\label{NL_model}
\end{equation}
where $f\left( .,.\right) $ is a known univariate function, $x_{t-1}$ is a $d$-dimensional integrated process of order one, $\theta _{0}$ and $\gamma _{0}$ are unknown parameters and $e_{t}$ is a martingale difference process.

To help ease the curse of dimensionality associated
with estimating multivariate non-linear models with integrated time series, we present an econometric model in which the multiple integrated regressors can be reduced to a univariate single-index form via a known univariate nonlinear function. This single-index component allows for either cointegrated predictors or non-cointegrated predictors. We develop a new estimation procedure for the model. We apply the proposed model to study stock return predictability and the main findings in chapter 2 include:
\begin{enumerate}
	\item We show, via a Monte Carlo experiment, that the new estimator has
	better finite sample properties than the standard nonlinear least-squares
	estimator. 
	\item Exploiting nonlinearities in the data can lead to improved forecast accuracy relative to the historical average when predicting stock returns. 
\end{enumerate}

\textcolor{blue}{expand findings; put models in overview}

The empirical results of nonlinear models proved that by considering the nonlinear relationship among variables, the out-of-sample performance can be improved significantly. However, the nonlinear model we propose above only include integrated regressors,  while in economic and financial
studies, the regressors are a mixture of stationary and non-stationary variables. For example,  some stationary predictors are also proved to be important, such as the regression residuals in the study of \cite{lettau2001consumption}. 

In addition, the nonlinear model fail to consider the lag variables, $y_{t-1}, y_{t-2}, \cdots, y_{t-p}$, which will lead to serially correlated residuals. Therefore, we further extend model \ref{NL_model} and propose the partially nonlinear model:

\begin{equation}
	y_{t} = \beta_0^{\prime} z_t + f\left( x_{t-1}^{\prime }\theta_0; \gamma_0\right) +e_{t},\ \ \
	t=2,...,T,  
	\label{PL model}
\end{equation}%
where $z_t = (y_{t-1}, \cdots, y_{t-p}, w_{t-1}^{\prime})^{\prime}$, in which $w_{t-1}$ is a vector of stationary predictors, $g\left( .,.\right) $ is a known univariate nonlinear function, $x_{t-1}$ is a $d$-dimensional integrated process of order one, $\theta _{0}$ and $\gamma _{0}$ are unknown parameters and $e_{t}$ is a martingale
difference process.

To estimate model (\ref{PL model}), we propose a novel 3-step estimation method in which $\beta$ will have a closed form solution while $\theta$ and $\gamma$ can be estimated by the method of nonlinear least squares or constrained nonlinear least squares. As we need to use an iterative procedure to apply the nonlinear least square estimation, we choose initial values calculated from Taylor expansion to start the procedure.

The findings of this chapter are summarized as follows:

\begin{enumerate}
    \item From the Monte Carlo simulation, we find that the constrained nonlinear least square estimators have good finite sample performances
    
    \item The Taylor initial values we calculate from linear regression help further improve the performance. 
    
    \item Model (\ref{PL model}) outperforms pure linear or nonlinear models, including autoregressive models and model (\ref{NL_model}).
    
    \item The partially nonlinear model we proposed also provide a better out-of-sample performance than historical average.
\end{enumerate}

\section{Overview of the Thesis}
The thesis contains 4 chapters. The contents of each chapter are described below.
\subsection*{Chapter 1: Introduction}

\subsection*{Chapter 2: Literature Review}

\subsection*{Chapter 3: Time-varying Modelling of Fertility and the Tax Benefits}

\subsection*{Chapter 4: Non-stationary Parametric Single-Index Predictive Models}

\subsection*{Chapter 5: Partially Nonlinear Single-Index Predictive Models}

\subsection*{Chapter 6: Conclusion}

\pagebreak

\section{Timetable}
\subsection{Statement of Progress}
Most of the materials for chapter 1 and 2 are already included in the separated chapters and papers. I will need to reorganize the materials I have and chapter 1 and 2 can be completed within a month. Chapter 3 and 4 are practically completed. For chapter 5, the simulations and out-of-sample prediction has been completed, we will need to add significance test results and the realised utility gains. This can also be finished in one month. The concluding chapter will be completed at the end.  

\subsection{Timetable}
\subsubsection{March 2022 - June 2022}
\begin{itemize}
    \item Test the significance of $R^{2}_{OOS}$ in the empirical study following \cite{clark2007approximately}。
    \item Calculate the realised utility gain following \cite{campbell2008predicting} and \cite{neely2014forecasting}.
    \item Completing Chapter 5 on the partially nonlinear model.
\end{itemize}
\subsubsection {July 2022 - August 2022}
\begin{itemize}
    \item Completing the introduction, literature review, and conclusion chapters and submit the thesis.
\end{itemize}

\subsection{Difficulties}


{\footnotesize
	\bibliographystyle{agsm}
	\bibliography{reference}
}
	
\end{document} 